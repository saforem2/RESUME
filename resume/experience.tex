%-------------------------------------------------------------------------------
%	SECTION TITLE
%-------------------------------------------------------------------------------
\vspace{-1.75ex}
\cvsection{Experience}

%-------------------------------------------------------------------------------
%	CONTENT
%-------------------------------------------------------------------------------
\begin{cventries}

%---------------------------------------------------------
\cventry
  {Graduate Research Fellow} % Job title
  {Argonne National Laboratory, Computational Sciences Division} % Organization
  {Lemont, IL} % Location
  {Summer 2018 - Summer 2019} % Date(s)
  {
      \begin{cvitems} % Description(s) of tasks/responsibilities
          \item {Software development focused on applying machine learning
                  models to help improve the efficiency of Hybrid Monte Carlo 
                simulations and their use in Lattice QCD.}
                  % \href{https://github.com/saforem2/l2hmc-qcd}{(source)}}
          \item {Built and deployed learning models using Tensorflow/Keras 
                  on some of the world's fastest supercomputers using 
                  % on Argonne's massively parallel, many-core supercomputer (Theta) using
                  state-of-the-art high-performance computing techniques.}
          \item {Developed a method for training Markov Chain Monte
                  Carlo kernels parameterized with deep neural networks that
                  shows promise in out performing traditional methods on a
                  variety of different models.}
              %     successfully outperforms traditional methods for a variety of
              % different models.}
      \end{cvitems}
  }
%---------------------------------------------------------

\cventry
  {Research Assistant} % Job title
  {University of Iowa, Department of Physics \& Astronomy} % Organization
  {Iowa City, IA} % Location
  {Spring 2016 - Fall 2016} % Date(s)
  {
      \begin{cvitems} % Description(s) of tasks/responsibilities
          \item {Software and hardware development for HaloSat, a
                  nanosatellite built with the goal of better understanding
                  the missing baryon problem.}
          % \item {Helped to design the event detection logic and was responsible
          %         for creating the interface necessary for digitizing and
          %         interpreting received signals.}
          % \item {Worked with a team to build the telemetry system that handled
          %         data transmission between the satellite and ground
          %         communications networks.}
          \item {Implemented a variety of in-flight optimization algorithms aimed at
                  maximizing the incoming X-ray signals (by minimizing
                  background noise) while in operation.}
      \end{cvitems}
  }
%---------------------------------------------------------

  \cventry
  {Research Assistant} % Job title 
  {University of Illinois, Center for Complex Systems Research} % Organization
  {Champaign, IL}
  {Spring 2011 - Spring 2015}
  {
      \begin{cvitems} 
          \item {Actively maintained the legacy code base (C++ / MATLAB) for our
                  research group and was in charge of quality analysis of new contributions.}
          \item {Constructed a model capable of describing the energy density and
                  self-discharge time of nanoscale capacitors.}
                  % These results
                  % helped to identify flaws in many of the previous models and
                  % allowed us to create a new design capable of a significantly
                  % longer self-discharge time, and consequently, a much larger
                  % maximum energy density.}
          \item {This work was submitted as a patent (pending) titled "Energy
                  Storage in Quantum Resonators", on which I was designated a
                  co-inventor together with my advisor Alfred H\"ubler.}
      \end{cvitems}
  }

\end{cventries}
          % \item {Wrote (and maintained) the majority of the molecular dynamics
          %         software responsible for carrying out simulations reliably,
          %         efficiently, and accurately.}
