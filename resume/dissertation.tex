%-------------------------------------------------------------------------------
%	SECTION TITLE
%-------------------------------------------------------------------------------
\vspace{-1.75ex}
\cvsection{Dissertation}

%-------------------------------------------------------------------------------
%	CONTENT
%-------------------------------------------------------------------------------

\begin{cventries}
%---------------------------------------------------------
  \cventry%
  {Supervisor: Yannick Meurice} % Job title
  {\href{https://ir.uiowa.edu/etd/6944/}{\textbf{Learning Better
  Physics: A Machine Learning Approach to Lattice Gauge Theory}}}
  % {University of Iowa, Department of Physics \& Astronomy} % Organization
  {University of Iowa} % Location
  {Spring 2016 -- Summer 2019} % Date(s)
  {
    \begin{cvitemscustom} % Description(s) of tasks/responsibilities
    \item{Developed new technique for applying renormalization group
      transformations to arbitrary 2D images.}
    \item{Designed and implemented a new algorithmic technique for Markov Chain
      Monte Carlo simulations that is able to generate gauge configurations.}
    \item {Carried out interdisciplinary research focused on applying ideas
        from machine learning and data science to simulations in high-energy
      physics.}
    \item{Discovered a new method for describing the phase transition in the
        2-dimensional Ising model by applying unsupervised learning techniques
      (PCA, k-means clustering) to Monte Carlo simulation images.}
    \item {Helped to create a new technique for implementing
        renormalization group transformations on arbitrary image
        sets, and explored potential applications in dynamic image
      analysis and action recognition.}
    \item {Worked with Tensorflow/Keras to construct convolutional
        neural networks capable of classifying configurations of the
      Ising model by temperature.}
    \item {Current work focuses on improving the efficiency of the Hybrid
        Monte Carlo algorithm by using neural networks to improve the
        quality of the sampler. These improvements have wide applications
      across a variety of industries.}
    \end{cvitemscustom}
  }
\end{cventries}
% same temperature). a collection of these
% images' covariance matrixkw
% covariance matrix formed from a collection of covariance
% and the physical observable used to describe the phase
% transition, specific heat capacity at constant
% volume)}
% and the physical observable (specific heat capacity) physically relevant quantity (the specific heat ca
% specific heat capacity.
% \item {Identified relationship between the leading principal component of
%         the covariance matrix obtained from configurations of the
%         two-dimensional Ising model represented as two-dimensional
%         greyscale images, and the specific heat capacity at constant
%         volume that describes the second degree phase transition.}
% This was done by identifying a
% relationship between the principal component of the covariance matrix
% obtained from equilibrium configurations represented as
% two-dimensional greyscale images, and the specific heat capacity at
% constant volume.
