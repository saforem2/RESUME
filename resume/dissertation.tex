%-------------------------------------------------------------------------------
%	SECTION TITLE
%-------------------------------------------------------------------------------
\vspace{-1.75ex}
\cvsection{Thesis Research}

%-------------------------------------------------------------------------------
%	CONTENT
%-------------------------------------------------------------------------------

\begin{cventries}
%---------------------------------------------------------
  \cventry
  {Research Assistant} % Job title
  {University of Iowa, Department of Physics \& Astronomy} % Organization
  {Iowa City, IA} % Location
  {Spring 2016 - Present} % Date(s)
  {
      \begin{cvitems} % Description(s) of tasks/responsibilities
          \item {Carried out interdisciplinary research focused on applying
                  ideas from machine learning and data science to simulations in
                  high-energy physics.}
                  % lattice gauge theory (LGT) and lattice quantum chromodynamics (LQCD).}
          \item {Discovered a new method for describing the phase transition in
                  the 2-dimensional Ising model by applying unsupervised
                  learning techniques (PCA, k-means clustering) to Monte Carlo
                  simulation images.}
          % \item {Used unsupervised learning techniques (PCA, k-means
          %         clustering) as an alternative method for learning about the
          %         phase transition of the .}
                  % By representing
                  % equilibrium configurations of the system as two-dimensional
                  % greyscale images, I was able to obtain a relationship between
                  % the first principal component of the covariance matrix
                  % associated with a collection of such images, and the specific
                  % heat capacity (the physical quantity used to describe the
                  % phase transition).}
          \item {Helped to create a new technique for implementing
                  renormalization group transformations on arbitrary image
                  sets, and explored potential applications in dynamic image
                  analysis and action recognition.}
          \item {Worked with Tensorflow/Keras to construct convolutional
                  neural networks capable of classifying configurations of the
                  Ising model by temperature.}
          % \item {Built and trained a multi-layer convolutional neural network
          %         in tensorflow used for classifying configurations of the Ising
          %         model by temperature.}
                  % By studying the classification
                  % accuracy as a function of temperature, the existence of a
                  % phase transition can be identified by where the network
                  % begins to fail (characterized by a sharp spike in the
                  % classification error).}
          \item {Current work focuses on improving the efficiency of the Hybrid
                  Monte Carlo algorithm by using neural networks to improve the
                  quality of the sampler. These improvements have wide applications
                  across a variety of industries.}
          % \item {Current work involves using deep feed-forward neural
          %         networks and restricted Boltzmann machines to help improve
          %         the efficiency of Hybrid Monte Carlo simulations in
          %         LQCD.}
      \end{cvitems}
  }
\end{cventries}




                  % same temperature). a collection of these
                  % images' covariance matrixkw
                  % covariance matrix formed from a collection of covariance
                  % and the physical observable used to describe the phase
                  % transition, specific heat capacity at constant
                  % volume)}
      % and the physical observable (specific heat capacity) physically relevant quantity (the specific heat ca
      % specific heat capacity.
      % \item {Identified relationship between the leading principal component of
      %         the covariance matrix obtained from configurations of the
      %         two-dimensional Ising model represented as two-dimensional
      %         greyscale images, and the specific heat capacity at constant
      %         volume that describes the second degree phase transition.}
          % This was done by identifying a
          % relationship between the principal component of the covariance matrix
          % obtained from equilibrium configurations represented as
          % two-dimensional greyscale images, and the specific heat capacity at
          % constant volume.
