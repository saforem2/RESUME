% LaTeX file for resume
% This file uses the resume document class (res.cls)

\documentclass{res}
%\usepackage{helvetica} % uses helvetica postscript font (download helvetica.sty)
%\usepackage{newcent}   % uses new century schoolbook postscript font
\setlength{\textheight}{9.5in} % increase text height to fit on 1-page
\newcommand\tab[1][5mm]{\hspace*{#1}}
\usepackage[top=0.75in,left=0.75in,right=1.5in,bottom=1in]{geometry}

\begin{document}

%\moveleft
\centerline{\large\bf Samuel A. Foreman}
%\moveright.5
\centerline{samuel-foreman@uiowa.edu}
% Draw a horizontal line the whole width of resume:
 \moveleft\hoffset\vbox{\hrule width\resumewidth height 1pt}\smallskip

%\address{\bf  PRESENT ADDRESS\\215 Iowa Ave. Apt 5\\Iowa City, IA 52240\\(630) 873-0945\\samuel-foreman@uiowa.edu}
%\address{\bf PERMANENT ADDRESS \\ 4004 N Lincoln St \\  Westmont, IL 60559\\  (630) 873-0945\\saforem2@illinois.edu}

\begin{resume}

\section{\underline{CURRENT INTERESTS}}
\vspace{1.5mm}
   I am currently seeking a position in applied physics research, particularly in hardware-software interfacing, physical modeling, and/or data analysis. My current research focuses on hardware development, namely FPGA (Verilog/HDL) programming, and telemetry development for a satellite to be launched by NASA in 2019. Moreover, I also have significant experience in quantum theory, the dynamics of nonlinear/complex systems, and a wide variety of modern experimental methods including but not limited to nanoscale electronics, integrated nanocircuits, nanolithography, and various AMO techniques.
\section{\underline{EDUCATION}}
\vspace{1.5mm}
   University of Illinois Urbana-Champaign\\
      \tab Bachelor of Science, Engineering Physics, May 2015   \\
 %   Major G.P.A. \textbf{3.45 / 4.0 }\\
 %   Cumulative G.P.A. \textbf{ 3.39 / 4.0} \\
   University of Illinois Urbana-Champaign\\
      \tab Bachelor of Science, Applied Mathematics, May 2015 \\
 %   Major G.P.A. \textbf{3.48 / 4.0 }\\
 %   Cumulative G.P.A. \textbf{ 3.39 / 4.0} \\
   University of Iowa\\
      \tab PhD. Student, Physics
 %     G.P.A. \textbf{3.44 / 4.0}\\
\section{\underline{RELEVANT EXPERIENCE}}
\vspace{1.5mm}
   \begin{tabbing}
   \hspace{1.75in}\= \hspace{2.75in}\= \kill % set up two tab positions
    {\bf Research Assistant} \> Department of Physics \& Astronomy\> Spring 2016 - Present\\
                             \> University of Iowa
   \end{tabbing}\vspace{-18pt}      % suppress blank line after tabbing
   Software and hardware development for HaloSat, with the goal of understanding the missing baryon problem. I've been responsible for developing the telemetry system for data transmission between the satellite and ground communications networks. In addition, I've worked on developing the operating system responsible for creating a peak detector circuit that is capable of interfacing with the various detectors in order to effectively discretize received signals. Further, I've developed optimization algorithms that work to maximize the incoming X-ray signals, while simultaneously minimizing background noise while our satellite is in operation. All research has been under the supervision and guidance of Professor Philip Kaaret.
   \vspace{-1mm}
   \begin{tabbing}
   \hspace{1.75in}\= \hspace{2.75in}\= \kill % set up two tab positions
    {\bf Research Assistant} \>Center for Complex Systems Research \> Spring 2011 - Fall 2015\\
                             \>University of Illinois
   \end{tabbing}\vspace{-18pt}      % suppress blank line after tabbing
    Modeling chaotic and Complex systems, both theoretically and experimentally using MATLAB, Python, and C/C++. I briefly investigated the existence of negative dimensional fractals by studying the propagation of heat in randomly distributed systems, in order to better understand the dynamics of nearest-neighbor dissipation. Following this, I worked on developing the theoretical framework behind Fowler-Nordheim emission in nanoscale capacitors. Using the predictions of this model, our team was successful in developing a digital quantum battery that effectively suppresses the Coulomb-Blockade, and which is able to retain energy densities that are orders of magnitude greater than current lighium-ion technologies. This research has been submitted as a patent with the United States Patent and Trademark Office, and has been submitted to the Journal of Vacuum Science \& Technology B. All research has been under the supervision and guidance of Professor Alfred H\"{u}bler.
%\begin{tabbing}
   %\hspace{2.3in}\= \hspace{2.6in}\= \kill % set up two tab positions
    %{\bf Beta Tester} \> Wolfram-Alpha \> Fall 2011 - Present\\
    %                      \>Champaign, IL
   %\end{tabbing}\vspace{-20pt}
  % 	Testing experimental features and reporting errors  for Wolfram Alpha. (Volunteer)

  	%\vspace{-5pt}

	 %\begin{tabbing}%
   %\hspace{2.3in}\= \hspace{2.6in}\= \kill % set up two tab positions
   %{\bf Teacher's Assistant}  \> Math \& Astronomy Departments \> Spring 2012/Fall 2013\\
   %                       \>University of Illinois, Champaign, IL
   %\end{tabbing}\vspace{-20pt}
   % Duties included grading assignments for both undergraduate astronomy and mathematics courses, as well as serving as a guide for students in the observatory.
\section{\underline{ADDITIONAL SKILLS}}
\vspace{1.5mm}
    Extensive knowledge of the Linux kernel, Python, shell scripting, C/C++, and Matlab. \\
    Familiarity with Verilog/FPGA programming, as well as SPI interfacing between microcontrollers.\\
    Moderate programming skills in Javascript, Ruby, XSPEC, and ROOT. \\

 %\section{HONORS AND AWARDS}
 %   Dean's List of Distinguished Students  \\
 %   Science Olympiad: Westmont High School  \\

% \section{EXTRACURRICULAR ACTIVITIES}
    %Epsilon Delta Sigma Public Relations Committee \\
    %Rensselaer Ski Club     \\
    %Bergen County Task Force Student Liaison 1986  \\
   % LEADD (Legislators and Educators Against Drunk Driving) Chairman
   %  1985-86  \\
  %  Youth Group - Temple Beth Or Activities Chairman 1985-86     %

\end{resume}

%%%%% UPDATED LINKEDIN SUMMARY %%%%%
%I am currently seeking a position in applied physics research, particularly hardware-software interfacing, physical modeling, and/or data analysis. My current research focuses on hardware development, including but not limited to FPGA programming, and telemetry development for a satellite to be launched by NASA in 2019. Moreover, I also have significant experience in quantum theory, the dynamics of nonlinear/complex systems, and a wide variety of modern experimental methods including nanoscale electronics, integrated nanocircuits, nanolithography, and various AMO techniques. In addition, I was a research assistant (4+ years) at the Center for Complex Systems Research at The University of Illinois at Urbana-Champaign, focusing primarily on the theoretical framework behind the ongoing development of digital quantum batteries, resulting in a patent (pending) and a publication in the Journal of Vacuum Science and Technology B.


\end{document}
